\PassOptionsToPackage{unicode=true}{hyperref} % options for packages loaded elsewhere
\PassOptionsToPackage{hyphens}{url}
%
\documentclass[]{article}
\usepackage{lmodern}
\usepackage{amssymb,amsmath}
\usepackage{ifxetex,ifluatex}
\usepackage{fixltx2e} % provides \textsubscript
\ifnum 0\ifxetex 1\fi\ifluatex 1\fi=0 % if pdftex
  \usepackage[T1]{fontenc}
  \usepackage[utf8]{inputenc}
  \usepackage{textcomp} % provides euro and other symbols
\else % if luatex or xelatex
  \usepackage{unicode-math}
  \defaultfontfeatures{Ligatures=TeX,Scale=MatchLowercase}
\fi
% use upquote if available, for straight quotes in verbatim environments
\IfFileExists{upquote.sty}{\usepackage{upquote}}{}
% use microtype if available
\IfFileExists{microtype.sty}{%
\usepackage[]{microtype}
\UseMicrotypeSet[protrusion]{basicmath} % disable protrusion for tt fonts
}{}
\IfFileExists{parskip.sty}{%
\usepackage{parskip}
}{% else
\setlength{\parindent}{0pt}
\setlength{\parskip}{6pt plus 2pt minus 1pt}
}
\usepackage{hyperref}
\hypersetup{
            pdftitle={Homework 4},
            pdfborder={0 0 0},
            breaklinks=true}
\urlstyle{same}  % don't use monospace font for urls
\usepackage[margin=1in]{geometry}
\usepackage{color}
\usepackage{fancyvrb}
\newcommand{\VerbBar}{|}
\newcommand{\VERB}{\Verb[commandchars=\\\{\}]}
\DefineVerbatimEnvironment{Highlighting}{Verbatim}{commandchars=\\\{\}}
% Add ',fontsize=\small' for more characters per line
\usepackage{framed}
\definecolor{shadecolor}{RGB}{248,248,248}
\newenvironment{Shaded}{\begin{snugshade}}{\end{snugshade}}
\newcommand{\AlertTok}[1]{\textcolor[rgb]{0.94,0.16,0.16}{#1}}
\newcommand{\AnnotationTok}[1]{\textcolor[rgb]{0.56,0.35,0.01}{\textbf{\textit{#1}}}}
\newcommand{\AttributeTok}[1]{\textcolor[rgb]{0.77,0.63,0.00}{#1}}
\newcommand{\BaseNTok}[1]{\textcolor[rgb]{0.00,0.00,0.81}{#1}}
\newcommand{\BuiltInTok}[1]{#1}
\newcommand{\CharTok}[1]{\textcolor[rgb]{0.31,0.60,0.02}{#1}}
\newcommand{\CommentTok}[1]{\textcolor[rgb]{0.56,0.35,0.01}{\textit{#1}}}
\newcommand{\CommentVarTok}[1]{\textcolor[rgb]{0.56,0.35,0.01}{\textbf{\textit{#1}}}}
\newcommand{\ConstantTok}[1]{\textcolor[rgb]{0.00,0.00,0.00}{#1}}
\newcommand{\ControlFlowTok}[1]{\textcolor[rgb]{0.13,0.29,0.53}{\textbf{#1}}}
\newcommand{\DataTypeTok}[1]{\textcolor[rgb]{0.13,0.29,0.53}{#1}}
\newcommand{\DecValTok}[1]{\textcolor[rgb]{0.00,0.00,0.81}{#1}}
\newcommand{\DocumentationTok}[1]{\textcolor[rgb]{0.56,0.35,0.01}{\textbf{\textit{#1}}}}
\newcommand{\ErrorTok}[1]{\textcolor[rgb]{0.64,0.00,0.00}{\textbf{#1}}}
\newcommand{\ExtensionTok}[1]{#1}
\newcommand{\FloatTok}[1]{\textcolor[rgb]{0.00,0.00,0.81}{#1}}
\newcommand{\FunctionTok}[1]{\textcolor[rgb]{0.00,0.00,0.00}{#1}}
\newcommand{\ImportTok}[1]{#1}
\newcommand{\InformationTok}[1]{\textcolor[rgb]{0.56,0.35,0.01}{\textbf{\textit{#1}}}}
\newcommand{\KeywordTok}[1]{\textcolor[rgb]{0.13,0.29,0.53}{\textbf{#1}}}
\newcommand{\NormalTok}[1]{#1}
\newcommand{\OperatorTok}[1]{\textcolor[rgb]{0.81,0.36,0.00}{\textbf{#1}}}
\newcommand{\OtherTok}[1]{\textcolor[rgb]{0.56,0.35,0.01}{#1}}
\newcommand{\PreprocessorTok}[1]{\textcolor[rgb]{0.56,0.35,0.01}{\textit{#1}}}
\newcommand{\RegionMarkerTok}[1]{#1}
\newcommand{\SpecialCharTok}[1]{\textcolor[rgb]{0.00,0.00,0.00}{#1}}
\newcommand{\SpecialStringTok}[1]{\textcolor[rgb]{0.31,0.60,0.02}{#1}}
\newcommand{\StringTok}[1]{\textcolor[rgb]{0.31,0.60,0.02}{#1}}
\newcommand{\VariableTok}[1]{\textcolor[rgb]{0.00,0.00,0.00}{#1}}
\newcommand{\VerbatimStringTok}[1]{\textcolor[rgb]{0.31,0.60,0.02}{#1}}
\newcommand{\WarningTok}[1]{\textcolor[rgb]{0.56,0.35,0.01}{\textbf{\textit{#1}}}}
\usepackage{graphicx,grffile}
\makeatletter
\def\maxwidth{\ifdim\Gin@nat@width>\linewidth\linewidth\else\Gin@nat@width\fi}
\def\maxheight{\ifdim\Gin@nat@height>\textheight\textheight\else\Gin@nat@height\fi}
\makeatother
% Scale images if necessary, so that they will not overflow the page
% margins by default, and it is still possible to overwrite the defaults
% using explicit options in \includegraphics[width, height, ...]{}
\setkeys{Gin}{width=\maxwidth,height=\maxheight,keepaspectratio}
\setlength{\emergencystretch}{3em}  % prevent overfull lines
\providecommand{\tightlist}{%
  \setlength{\itemsep}{0pt}\setlength{\parskip}{0pt}}
\setcounter{secnumdepth}{0}
% Redefines (sub)paragraphs to behave more like sections
\ifx\paragraph\undefined\else
\let\oldparagraph\paragraph
\renewcommand{\paragraph}[1]{\oldparagraph{#1}\mbox{}}
\fi
\ifx\subparagraph\undefined\else
\let\oldsubparagraph\subparagraph
\renewcommand{\subparagraph}[1]{\oldsubparagraph{#1}\mbox{}}
\fi

% set default figure placement to htbp
\makeatletter
\def\fps@figure{htbp}
\makeatother

\usepackage{etoolbox}
\makeatletter
\providecommand{\subtitle}[1]{% add subtitle to \maketitle
  \apptocmd{\@title}{\par {\large #1 \par}}{}{}
}
\makeatother

\title{Homework 4}
\providecommand{\subtitle}[1]{}
\subtitle{Due October 14, 2020}
\author{}
\date{\vspace{-2.5em}2020-09-08}

\begin{document}
\maketitle

For each assignment, turn in by the due date/time. Late assignments must
be arranged prior to submission. In every case, assignments are to be
typed neatly using proper English in Markdown.

The last couple of weeks, we spoke about vector/matrix operations in R,
discussed how the apply family of functions can assist in row/column
operations, and how parallel computing in R is enabled. Combining this
with previous topics, we can write functions using our Good Programming
Practices style and adhere to Reproducible Research principles to create
fully functional, redable and reproducible code based analysis in R. In
this homework, we will put this all together and actually analyze some
data. Remember to adhere to both Reproducible Research and Good
Programming Practices, ie describe what you are doing and comment/indent
code where necessary.

R Vector/matrix manipulations and math, speed considerations R's Apply
family of functions Parallel computing in R, foreach and dopar

\hypertarget{problem-2-using-the-dual-nature-to-our-advantage}{%
\subsection{Problem 2: Using the dual nature to our
advantage}\label{problem-2-using-the-dual-nature-to-our-advantage}}

Sometimes using a mixture of true matrix math plus component operations
cleans up our code giving better readibility. Suppose we wanted to form
the following computation:

\begin{itemize}
    \item $while(abs(\Theta_0^{i}-\Theta_0^{i-1}) \text{ AND } abs(\Theta_1^{i}-\Theta_1^{i-1}) > tolerance) \text{ \{ }$
    \begin{eqnarray*}
        \Theta_0^i &=& \Theta_0^{i-1} - \alpha\frac{1}{m}\sum_{i=1}^{m} (h_0(x_i) -y_i)  \\
        \Theta_1^i &=& \Theta_1^{i-1} - \alpha\frac{1}{m}\sum_{i=1}^{m} ((h_0(x_i) -y_i)x_i) 
    \end{eqnarray*}
    $\text{ \} }$
\end{itemize}

Where \(h_0(x) = \Theta_0 + \Theta_1x\).

Given \(\mathbf{X}\) and \(\vec{h}\) below, implement the above
algorithm and compare the results with
lm(h\textasciitilde{}0+\(\mathbf{X}\)). State the tolerance used and the
step size, \(\alpha\).

\begin{Shaded}
\begin{Highlighting}[]
    \KeywordTok{set.seed}\NormalTok{(}\DecValTok{1256}\NormalTok{)}
\NormalTok{    theta <-}\StringTok{ }\KeywordTok{as.matrix}\NormalTok{(}\KeywordTok{c}\NormalTok{(}\DecValTok{1}\NormalTok{,}\DecValTok{2}\NormalTok{),}\DataTypeTok{nrow=}\DecValTok{2}\NormalTok{)}
\NormalTok{    X <-}\StringTok{ }\KeywordTok{cbind}\NormalTok{(}\DecValTok{1}\NormalTok{,}\KeywordTok{rep}\NormalTok{(}\DecValTok{1}\OperatorTok{:}\DecValTok{10}\NormalTok{,}\DecValTok{10}\NormalTok{))}
\NormalTok{    h <-}\StringTok{ }\NormalTok{X}\OperatorTok\NormalTok{theta}\OperatorTok{+}\KeywordTok{rnorm}\NormalTok{(}\DecValTok{100}\NormalTok{,}\DecValTok{0}\NormalTok{,}\FloatTok{0.2}\NormalTok{)}
\end{Highlighting}
\end{Shaded}

\hypertarget{problem-3}{%
\subsection{Problem 3}\label{problem-3}}

The above algorithm is called Gradient Descent. This algorithm, like
Newton's method, has ``hyperparameters'' that are determined outside the
algorithm and there are no set rules for determing what settings to use.
For gradient descent, you need to set a start value, a step size and
tolerance.

\hypertarget{part-a.-using-a-step-size-of-1e-7-and-tolerance-of-1e-9-try-10000-different-combinations-of-start-values-for-beta_0-and-beta_1-across-the-range-of-possible-betas--1-from-true-determined-in-problem-2-making-sure-to-take-advantages-of-parallel-computing-opportunities.-in-my-try-at-this-i-found-starting-close-to-true-took-1.1m-iterations-so-set-a-stopping-rule-for-5m.-report-the-min-and-max-number-of-iterations-along-with-the-starting-values-for-those-cases.-also-report-the-average-and-stdev-obtained-across-all-10000-betas.}{%
\subsubsection{\texorpdfstring{Part a. Using a step size of \(1e^{-7}\)
and tolerance of \(1e^{-9}\), try 10000 different combinations of start
values for \(\beta_0\) and \(\beta_1\) across the range of possible
\(\beta\)'s +/-1 from true determined in Problem 2, making sure to take
advantages of parallel computing opportunities. In my try at this, I
found starting close to true took 1.1M iterations, so set a stopping
rule for 5M. Report the min and max number of iterations along with the
starting values for those cases. Also report the average and stdev
obtained across all 10000
\(\beta\)'s.}{Part a. Using a step size of 1e\^{}\{-7\} and tolerance of 1e\^{}\{-9\}, try 10000 different combinations of start values for \textbackslash{}beta\_0 and \textbackslash{}beta\_1 across the range of possible \textbackslash{}beta's +/-1 from true determined in Problem 2, making sure to take advantages of parallel computing opportunities. In my try at this, I found starting close to true took 1.1M iterations, so set a stopping rule for 5M. Report the min and max number of iterations along with the starting values for those cases. Also report the average and stdev obtained across all 10000 \textbackslash{}beta's.}}\label{part-a.-using-a-step-size-of-1e-7-and-tolerance-of-1e-9-try-10000-different-combinations-of-start-values-for-beta_0-and-beta_1-across-the-range-of-possible-betas--1-from-true-determined-in-problem-2-making-sure-to-take-advantages-of-parallel-computing-opportunities.-in-my-try-at-this-i-found-starting-close-to-true-took-1.1m-iterations-so-set-a-stopping-rule-for-5m.-report-the-min-and-max-number-of-iterations-along-with-the-starting-values-for-those-cases.-also-report-the-average-and-stdev-obtained-across-all-10000-betas.}}

\hypertarget{part-b.-what-if-you-were-to-change-the-stopping-rule-to-include-our-knowledge-of-the-true-value-is-this-a-good-way-to-run-this-algorithm-what-is-a-potential-problem}{%
\subsubsection{Part b. What if you were to change the stopping rule to
include our knowledge of the true value? Is this a good way to run this
algorithm? What is a potential
problem?}\label{part-b.-what-if-you-were-to-change-the-stopping-rule-to-include-our-knowledge-of-the-true-value-is-this-a-good-way-to-run-this-algorithm-what-is-a-potential-problem}}

\hypertarget{part-c.-what-are-your-thoughts-on-this-algorithm}{%
\subsubsection{Part c. What are your thoughts on this
algorithm?}\label{part-c.-what-are-your-thoughts-on-this-algorithm}}

\hypertarget{problem-4-inverting-matrices}{%
\subsection{Problem 4: Inverting
matrices}\label{problem-4-inverting-matrices}}

Ok, so John Cook makes some good points, but if you want to do:

\begin{equation*}
\hat\beta = (X'X)^{-1}X'\underline{y}
\end{equation*}

what are you to do?? Can you explain what is going on?

\hypertarget{problem-5-need-for-speed-challenge}{%
\subsection{Problem 5: Need for speed
challenge}\label{problem-5-need-for-speed-challenge}}

In this problem, we are looking to compute the following:

\begin{equation}
y = p + A B^{-1} (q - r)
\end{equation}

Where A, B, p, q and r are formed by:

\begin{Shaded}
\begin{Highlighting}[]
    \KeywordTok{set.seed}\NormalTok{(}\DecValTok{12456}\NormalTok{) }
    
\NormalTok{    G <-}\StringTok{ }\KeywordTok{matrix}\NormalTok{(}\KeywordTok{sample}\NormalTok{(}\KeywordTok{c}\NormalTok{(}\DecValTok{0}\NormalTok{,}\FloatTok{0.5}\NormalTok{,}\DecValTok{1}\NormalTok{),}\DataTypeTok{size=}\DecValTok{16000}\NormalTok{,}\DataTypeTok{replace=}\NormalTok{T),}\DataTypeTok{ncol=}\DecValTok{10}\NormalTok{)}
\NormalTok{    R <-}\StringTok{ }\KeywordTok{cor}\NormalTok{(G) }\CommentTok{# R: 10 * 10 correlation matrix of G}
\NormalTok{    C <-}\StringTok{ }\KeywordTok{kronecker}\NormalTok{(R, }\KeywordTok{diag}\NormalTok{(}\DecValTok{1600}\NormalTok{)) }\CommentTok{# C is a 16000 * 16000 block diagonal matrix}
\NormalTok{    id <-}\StringTok{ }\KeywordTok{sample}\NormalTok{(}\DecValTok{1}\OperatorTok{:}\DecValTok{16000}\NormalTok{,}\DataTypeTok{size=}\DecValTok{932}\NormalTok{,}\DataTypeTok{replace=}\NormalTok{F)}
\NormalTok{    q <-}\StringTok{ }\KeywordTok{sample}\NormalTok{(}\KeywordTok{c}\NormalTok{(}\DecValTok{0}\NormalTok{,}\FloatTok{0.5}\NormalTok{,}\DecValTok{1}\NormalTok{),}\DataTypeTok{size=}\DecValTok{15068}\NormalTok{,}\DataTypeTok{replace=}\NormalTok{T) }\CommentTok{# vector of length 15068}
\NormalTok{    A <-}\StringTok{ }\NormalTok{C[id, }\OperatorTok{-}\NormalTok{id] }\CommentTok{# matrix of dimension 932 * 15068}
\NormalTok{    B <-}\StringTok{ }\NormalTok{C[}\OperatorTok{-}\NormalTok{id, }\OperatorTok{-}\NormalTok{id] }\CommentTok{# matrix of dimension 15068 * 15068}
\NormalTok{    p <-}\StringTok{ }\KeywordTok{runif}\NormalTok{(}\DecValTok{932}\NormalTok{,}\DecValTok{0}\NormalTok{,}\DecValTok{1}\NormalTok{)}
\NormalTok{    r <-}\StringTok{ }\KeywordTok{runif}\NormalTok{(}\DecValTok{15068}\NormalTok{,}\DecValTok{0}\NormalTok{,}\DecValTok{1}\NormalTok{)}
\NormalTok{    C<-}\OtherTok{NULL} \CommentTok{#save some memory space}
\end{Highlighting}
\end{Shaded}

Part a.

How large (bytes) are A and B? Without any optimization tricks, how long
does the it take to calculate y?

Part b.

How would you break apart this compute, i.e., what order of operations
would make sense? Are there any mathmatical simplifications you can
make? Is there anything about the vectors or matrices we might take
advantage of?

Part c.

Use ANY means (ANY package, ANY trick, etc) necessary to compute the
above, fast. Wrap your code in ``system.time(\{\})'', everything you do
past assignment ``C \textless{}- NULL''.

\hypertarget{problem-3-1}{%
\subsection{Problem 3}\label{problem-3-1}}

\begin{enumerate}
\def\labelenumi{\alph{enumi}.}
\item
  Create a function that computes the proportion of successes in a
  vector. Use good programming practices.
\item
  Create a matrix to simulate 10 flips of a coin with varying degrees of
  ``fairness'' (columns = probability) as follows:
\end{enumerate}

\begin{Shaded}
\begin{Highlighting}[]
    \KeywordTok{set.seed}\NormalTok{(}\DecValTok{12345}\NormalTok{)}
\NormalTok{    P4b_data <-}\StringTok{ }\KeywordTok{matrix}\NormalTok{(}\KeywordTok{rbinom}\NormalTok{(}\DecValTok{10}\NormalTok{, }\DecValTok{1}\NormalTok{, }\DataTypeTok{prob =}\NormalTok{ (}\DecValTok{31}\OperatorTok{:}\DecValTok{40}\NormalTok{)}\OperatorTok{/}\DecValTok{100}\NormalTok{), }\DataTypeTok{nrow =} \DecValTok{10}\NormalTok{, }\DataTypeTok{ncol =} \DecValTok{10}\NormalTok{, }\DataTypeTok{byrow =} \OtherTok{FALSE}\NormalTok{)}
\end{Highlighting}
\end{Shaded}

\begin{enumerate}
\def\labelenumi{\alph{enumi}.}
\setcounter{enumi}{2}
\item
  Use your function in conjunction with apply to compute the proportion
  of success in P4b\_data by column and then by row. What do you
  observe? What is going on?
\item
  You are to fix the above matrix by creating a function whose input is
  a probability and output is a vector whose elements are the outcomes
  of 10 flips of a coin. Now create a vector of the desired
  probabilities. Using the appropriate apply family function, create the
  matrix we really wanted above. Prove this has worked by using the
  function created in part a to compute and tabulate the appropriate
  marginal successes.
\end{enumerate}

\hypertarget{problem-4}{%
\subsection{Problem 4}\label{problem-4}}

In Homework 4, we had a dataset we were to compute some summary
statistics from. The description of the data was given as ``a dataset
which has multiple repeated measurements from two devices by thirteen
Observers''. Where the device measurements were in columns ``dev1'' and
``dev2''. Reimport that dataset, change the names of ``dev1'' and
``dev2'' to x and y and do the following:

\begin{enumerate}
  \item create a function that accepts a dataframe of values, title, and x/y labels and creates a scatter plot
  \item use this function to create:
  \begin{enumerate}
    \item a single scatter plot of the entire dataset
    \item a seperate scatter plot for each observer (using the apply function)
  \end{enumerate}
\end{enumerate}

\hypertarget{problem-5}{%
\subsection{Problem 5}\label{problem-5}}

Our ultimate goal in this problem is to create an annotated map of the
US. I am giving you the code to create said map, you will need to
customize it to include the annotations.

Part a. Get and import a database of US cities and states. Here is some
R code to help:

\begin{Shaded}
\begin{Highlighting}[]
    \CommentTok{#we are grabbing a SQL set from here}
    \CommentTok{# http://www.farinspace.com/wp-content/uploads/us_cities_and_states.zip}

    \CommentTok{#download the files, looks like it is a .zip}
    \KeywordTok{library}\NormalTok{(downloader)}
    \KeywordTok{download}\NormalTok{(}\StringTok{"http://www.farinspace.com/wp-content/uploads/us_cities_and_states.zip"}\NormalTok{,}\DataTypeTok{dest=}\StringTok{"us_cities_states.zip"}\NormalTok{)}
    \KeywordTok{unzip}\NormalTok{(}\StringTok{"us_cities_states.zip"}\NormalTok{, }\DataTypeTok{exdir=}\StringTok{"./"}\NormalTok{)}
    
    \CommentTok{#read in data, looks like sql dump, blah}
    \KeywordTok{library}\NormalTok{(data.table)}
\NormalTok{    states <-}\StringTok{ }\KeywordTok{fread}\NormalTok{(}\DataTypeTok{input =} \StringTok{"./us_cities_and_states/states.sql"}\NormalTok{,}\DataTypeTok{skip =} \DecValTok{23}\NormalTok{,}\DataTypeTok{sep =} \StringTok{"'"}\NormalTok{, }\DataTypeTok{sep2 =} \StringTok{","}\NormalTok{, }\DataTypeTok{header =}\NormalTok{ F, }\DataTypeTok{select =} \KeywordTok{c}\NormalTok{(}\DecValTok{2}\NormalTok{,}\DecValTok{4}\NormalTok{))}
    \CommentTok{### YOU do the CITIES}
    \CommentTok{### I suggest the cities_extended.sql may have everything you need}
    \CommentTok{### can you figure out how to limit this to the 50?}
\end{Highlighting}
\end{Shaded}

Part b. Create a summary table of the number of cities included by
state.

Part c. Create a function that counts the number of occurances of a
letter in a string. The input to the function should be ``letter'' and
``state\_name''. The output should be a scalar with the count for that
letter.

Create a for loop to loop through the state names imported in part a.
Inside the for loop, use an apply family function to iterate across a
vector of letters and collect the occurance count as a vector.

\begin{Shaded}
\begin{Highlighting}[]
    \CommentTok{##pseudo code}
\NormalTok{    letter_count <-}\StringTok{ }\KeywordTok{data.frame}\NormalTok{(}\KeywordTok{matrix}\NormalTok{(}\OtherTok{NA}\NormalTok{,}\DataTypeTok{nrow=}\DecValTok{50}\NormalTok{, }\DataTypeTok{ncol=}\DecValTok{26}\NormalTok{))}
\NormalTok{    getCount <-}\StringTok{ }\ControlFlowTok{function}\NormalTok{(what args)\{}
\NormalTok{        temp <-}\StringTok{ }\KeywordTok{strsplit}\NormalTok{(state_name)}
        \CommentTok{# how to count??}
        \KeywordTok{return}\NormalTok{(count)}
\NormalTok{    \}}
    \ControlFlowTok{for}\NormalTok{(i }\ControlFlowTok{in} \DecValTok{1}\OperatorTok{:}\DecValTok{50}\NormalTok{)\{}
\NormalTok{        letter_count[i,] <-}\StringTok{ }\NormalTok{xx}\OperatorTok{-}\KeywordTok{apply}\NormalTok{(args)}
\NormalTok{    \}}
\end{Highlighting}
\end{Shaded}

Part d.

Create 2 maps to finalize this. Map 1 should be colored by count of
cities on our list within the state. Map 2 should highlight only those
states that have more than 3 occurances of ANY letter in thier name.

Quick and not so dirty map:

\begin{Shaded}
\begin{Highlighting}[]
    \CommentTok{#https://cran.r-project.org/web/packages/fiftystater/vignettes/fiftystater.html}
    \KeywordTok{library}\NormalTok{(ggplot2)}
    \KeywordTok{library}\NormalTok{(fiftystater)}
    
    \KeywordTok{data}\NormalTok{(}\StringTok{"fifty_states"}\NormalTok{) }\CommentTok{# this line is optional due to lazy data loading}
\NormalTok{    crimes <-}\StringTok{ }\KeywordTok{data.frame}\NormalTok{(}\DataTypeTok{state =} \KeywordTok{tolower}\NormalTok{(}\KeywordTok{rownames}\NormalTok{(USArrests)), USArrests)}
    \CommentTok{# map_id creates the aesthetic mapping to the state name column in your data}
\NormalTok{    p <-}\StringTok{ }\KeywordTok{ggplot}\NormalTok{(crimes, }\KeywordTok{aes}\NormalTok{(}\DataTypeTok{map_id =}\NormalTok{ state)) }\OperatorTok{+}\StringTok{ }
\StringTok{      }\CommentTok{# map points to the fifty_states shape data}
\StringTok{      }\KeywordTok{geom_map}\NormalTok{(}\KeywordTok{aes}\NormalTok{(}\DataTypeTok{fill =}\NormalTok{ Assault), }\DataTypeTok{map =}\NormalTok{ fifty_states) }\OperatorTok{+}\StringTok{ }
\StringTok{      }\KeywordTok{expand_limits}\NormalTok{(}\DataTypeTok{x =}\NormalTok{ fifty_states}\OperatorTok{$}\NormalTok{long, }\DataTypeTok{y =}\NormalTok{ fifty_states}\OperatorTok{$}\NormalTok{lat) }\OperatorTok{+}
\StringTok{      }\KeywordTok{coord_map}\NormalTok{() }\OperatorTok{+}
\StringTok{      }\KeywordTok{scale_x_continuous}\NormalTok{(}\DataTypeTok{breaks =} \OtherTok{NULL}\NormalTok{) }\OperatorTok{+}\StringTok{ }
\StringTok{      }\KeywordTok{scale_y_continuous}\NormalTok{(}\DataTypeTok{breaks =} \OtherTok{NULL}\NormalTok{) }\OperatorTok{+}
\StringTok{      }\KeywordTok{labs}\NormalTok{(}\DataTypeTok{x =} \StringTok{""}\NormalTok{, }\DataTypeTok{y =} \StringTok{""}\NormalTok{) }\OperatorTok{+}
\StringTok{      }\KeywordTok{theme}\NormalTok{(}\DataTypeTok{legend.position =} \StringTok{"bottom"}\NormalTok{, }
            \DataTypeTok{panel.background =} \KeywordTok{element_blank}\NormalTok{())}
    
\NormalTok{    p}
    \CommentTok{#ggsave(plot = p, file = "HW6_Problem6_Plot_Settlage.pdf")}
\end{Highlighting}
\end{Shaded}

\hypertarget{problem-2}{%
\subsection{Problem 2}\label{problem-2}}

Bootstrapping

Recall the sensory data from five operators:\\
\url{http://www2.isye.gatech.edu/~jeffwu/wuhamadabook/data/Sensory.dat}

Sometimes, you really want more data to do the desired analysis, but
going back to the ``field'' is often not an option. An often used method
is bootstrapping. Check out the second answer here for a really nice and
detailed description of bootstrapping:
\url{https://stats.stackexchange.com/questions/316483/manually-bootstrapping-linear-regression-in-r}.

What we want to do is bootstrap the Sensory data to get non-parametric
estimates of the parameters. Assume that we can neglect item in the
analysis such that we are really only interested in a linear model
lm(y\textasciitilde{}operator).

\hypertarget{part-a.-first-the-question-asked-in-the-stackexchange-was-why-is-the-supplied-code-not-working.-this-question-was-actually-never-answered.-what-is-the-problem-with-the-code-if-you-want-to-duplicate-the-code-to-test-it-use-the-quantreg-package-to-get-the-data.}{%
\subsubsection{Part a. First, the question asked in the stackexchange
was why is the supplied code not working. This question was actually
never answered. What is the problem with the code? If you want to
duplicate the code to test it, use the quantreg package to get the
data.}\label{part-a.-first-the-question-asked-in-the-stackexchange-was-why-is-the-supplied-code-not-working.-this-question-was-actually-never-answered.-what-is-the-problem-with-the-code-if-you-want-to-duplicate-the-code-to-test-it-use-the-quantreg-package-to-get-the-data.}}

\hypertarget{part-b.-bootstrap-the-analysis-to-get-the-parameter-estimates-using-100-bootstrapped-samples.-make-sure-to-use-system.time-to-get-total-time-for-the-analysis.-you-should-probably-make-sure-the-samples-are-balanced-across-operators-ie-each-sample-draws-for-each-operator.}{%
\subsubsection{Part b. Bootstrap the analysis to get the parameter
estimates using 100 bootstrapped samples. Make sure to use system.time
to get total time for the analysis. You should probably make sure the
samples are balanced across operators, ie each sample draws for each
operator.}\label{part-b.-bootstrap-the-analysis-to-get-the-parameter-estimates-using-100-bootstrapped-samples.-make-sure-to-use-system.time-to-get-total-time-for-the-analysis.-you-should-probably-make-sure-the-samples-are-balanced-across-operators-ie-each-sample-draws-for-each-operator.}}

\hypertarget{part-c.-redo-the-last-problem-but-run-the-bootstraps-in-parallel-cl---makecluster8-dont-forget-to-stopclustercl.-why-can-you-do-this-make-sure-to-use-system.time-to-get-total-time-for-the-analysis.}{%
\subsubsection{\texorpdfstring{Part c. Redo the last problem but run the
bootstraps in parallel (\texttt{cl\ \textless{}-\ makeCluster(8)}),
don't forget to \texttt{stopCluster(cl)}). Why can you do this? Make
sure to use system.time to get total time for the
analysis.}{Part c. Redo the last problem but run the bootstraps in parallel (cl \textless{}- makeCluster(8)), don't forget to stopCluster(cl)). Why can you do this? Make sure to use system.time to get total time for the analysis.}}\label{part-c.-redo-the-last-problem-but-run-the-bootstraps-in-parallel-cl---makecluster8-dont-forget-to-stopclustercl.-why-can-you-do-this-make-sure-to-use-system.time-to-get-total-time-for-the-analysis.}}

Create a single table summarizing the results and timing from part a and
b. What are your thoughts?

\hypertarget{problem-3-2}{%
\subsection{Problem 3}\label{problem-3-2}}

Newton's method gives an answer for a root. To find multiple roots, you
need to try different starting values. There is no guarantee for what
start will give a specific root, so you simply need to try multiple.
From the plot of the function in HW4, problem 8, how many roots are
there?

Create a vector (\texttt{length.out=1000}) as a ``grid'' covering all
the roots and extending +/-1 to either end.

\hypertarget{part-a.-using-one-of-the-apply-functions-find-the-roots-noting-the-time-it-takes-to-run-the-apply-function.}{%
\subsubsection{Part a. Using one of the apply functions, find the roots
noting the time it takes to run the apply
function.}\label{part-a.-using-one-of-the-apply-functions-find-the-roots-noting-the-time-it-takes-to-run-the-apply-function.}}

\hypertarget{part-b.-repeat-the-apply-command-using-the-equivelant-parapply-command.-use-8-workers.-cl---makecluster8.}{%
\subsubsection{\texorpdfstring{Part b. Repeat the apply command using
the equivelant parApply command. Use 8 workers.
\texttt{cl\ \textless{}-\ makeCluster(8)}.}{Part b. Repeat the apply command using the equivelant parApply command. Use 8 workers. cl \textless{}- makeCluster(8).}}\label{part-b.-repeat-the-apply-command-using-the-equivelant-parapply-command.-use-8-workers.-cl---makecluster8.}}

Create a table summarizing the roots and timing from both parts a and b.
What are your thoughts?

\hypertarget{problem-9}{%
\subsection{Problem 9}\label{problem-9}}

Finish this homework by pushing your changes to your repo.

\textbf{Only submit the .Rmd and .pdf solution files. Names should be
formatted HW4\_pid.Rmd and HW4\_pid.pdf}

\end{document}
