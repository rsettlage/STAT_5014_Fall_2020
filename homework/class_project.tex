\PassOptionsToPackage{unicode=true}{hyperref} % options for packages loaded elsewhere
\PassOptionsToPackage{hyphens}{url}
%
\documentclass[]{article}
\usepackage{lmodern}
\usepackage{amssymb,amsmath}
\usepackage{ifxetex,ifluatex}
\usepackage{fixltx2e} % provides \textsubscript
\ifnum 0\ifxetex 1\fi\ifluatex 1\fi=0 % if pdftex
  \usepackage[T1]{fontenc}
  \usepackage[utf8]{inputenc}
  \usepackage{textcomp} % provides euro and other symbols
\else % if luatex or xelatex
  \usepackage{unicode-math}
  \defaultfontfeatures{Ligatures=TeX,Scale=MatchLowercase}
\fi
% use upquote if available, for straight quotes in verbatim environments
\IfFileExists{upquote.sty}{\usepackage{upquote}}{}
% use microtype if available
\IfFileExists{microtype.sty}{%
\usepackage[]{microtype}
\UseMicrotypeSet[protrusion]{basicmath} % disable protrusion for tt fonts
}{}
\IfFileExists{parskip.sty}{%
\usepackage{parskip}
}{% else
\setlength{\parindent}{0pt}
\setlength{\parskip}{6pt plus 2pt minus 1pt}
}
\usepackage{hyperref}
\hypersetup{
            pdftitle={Class project},
            pdfborder={0 0 0},
            breaklinks=true}
\urlstyle{same}  % don't use monospace font for urls
\usepackage[margin=1in]{geometry}
\usepackage{graphicx,grffile}
\makeatletter
\def\maxwidth{\ifdim\Gin@nat@width>\linewidth\linewidth\else\Gin@nat@width\fi}
\def\maxheight{\ifdim\Gin@nat@height>\textheight\textheight\else\Gin@nat@height\fi}
\makeatother
% Scale images if necessary, so that they will not overflow the page
% margins by default, and it is still possible to overwrite the defaults
% using explicit options in \includegraphics[width, height, ...]{}
\setkeys{Gin}{width=\maxwidth,height=\maxheight,keepaspectratio}
\setlength{\emergencystretch}{3em}  % prevent overfull lines
\providecommand{\tightlist}{%
  \setlength{\itemsep}{0pt}\setlength{\parskip}{0pt}}
\setcounter{secnumdepth}{0}
% Redefines (sub)paragraphs to behave more like sections
\ifx\paragraph\undefined\else
\let\oldparagraph\paragraph
\renewcommand{\paragraph}[1]{\oldparagraph{#1}\mbox{}}
\fi
\ifx\subparagraph\undefined\else
\let\oldsubparagraph\subparagraph
\renewcommand{\subparagraph}[1]{\oldsubparagraph{#1}\mbox{}}
\fi

% set default figure placement to htbp
\makeatletter
\def\fps@figure{htbp}
\makeatother

\usepackage{etoolbox}
\makeatletter
\providecommand{\subtitle}[1]{% add subtitle to \maketitle
  \apptocmd{\@title}{\par {\large #1 \par}}{}{}
}
\makeatother

\title{Class project}
\providecommand{\subtitle}[1]{}
\subtitle{Due December 9, 2020}
\author{}
\date{\vspace{-2.5em}2020-09-08}

\begin{document}
\maketitle

For each assignment, turn in by the due date/time. Late assignments must
be arranged prior to submission. In every case, assignments are to be
typed neatly using proper English in Markdown.

In this class, we have talked through many aspects of analysis in R. To
guide the tour through R, we introduced tutorials via Swirl or
Rstudio.cloud Primers. We added to this topics on Reproducible Research,
version control, and Good Programming Practices. In this project, we
need to tie it all together. Here, we will work in small teams to join
the data revolution. If nothing else, 2020 has been the year of the Data
Scientist. Between Covid-19 data, population income or other social
data, and election data, we have been locked at home with nothing
between us and a fun and perhaps informative data analysis except a
keyboard. In this project, the challenge will be to choose a large and
multivariate dataset, pull the data into R, and them munge it as
appropriate into a tidy dataset. From here, do something interesting
with it!

We will work in small teams, determined by class size. Choose one of the
three topics listed below, alter to fit your interest. Submit a project
proposal for approval. This should be submitted through a new GitHub
repo with all team members and myself listed as collaborators.

\hypertarget{project-1-covid-19}{%
\subsection{Project 1: COVID-19}\label{project-1-covid-19}}

We are awash in case load data. We also have data on hospital beds, ICU
beds, first responder counts, population demographics, etc. Create a
dashboard combining COVID case load data with some other demographic to
create an informative graphic. The graphic should contain a time
component, should allow fitering in some interesting way and create
meaningful summary statistics. Here are a couple of possible data
sources:

\url{https://covidatlas.com/data}\\
\url{https://hifld-geoplatform.opendata.arcgis.com/datasets/hospitals}

\hypertarget{project-2-social-disparity}{%
\subsection{Project 2: Social
disparity}\label{project-2-social-disparity}}

There are a ton of reports of social disparity in the U.S. Create a
dashboard combining racial population statistics with other data such as
income, health or other factor. This dashboard could include a time
component and should include a spatial component. Ideally, it would
involve a map and plots of the data. Question: does the time series data
accurately depict what the underlying data show? Are there other
features to the data that might suggest other trends that might be
interesting to follow up on? Here are a couple of papers and data
sources:

\url{https://www.racialequitytools.org/resourcefiles/lui.pdf}\\
\url{https://www.federalreserve.gov/econres/notes/feds-notes/recent-trends-in-wealth-holding-by-race-and-ethnicity-evidence-from-the-survey-of-consumer-finances-20170927.htm}~\\
\url{https://www.pgpf.org/blog/2019/10/income-and-wealth-in-the-united-states-an-overview-of-data}~\\
\url{https://www.census.gov/data/tables/time-series/demo/income-poverty/cps-pinc/pinc-01.html}

\hypertarget{project-3-election-2020}{%
\subsection{Project 3: Election 2020}\label{project-3-election-2020}}

This is an Presidential election year in the U.S. Here, you should
create a dashboard of political interest. Ideally, the dashboard will
contain a map and summary plots of the data. What data can you find that
might suggest one candidate over the other, locally, regionally,
nationally? Is there a time component to the data showing enthusiams
change? Is there social data available that might show trends? A few of
the many possible data sources:

Twitter:
\url{https://towardsdatascience.com/understanding-political-twitter-ce3476a38377}\\
\url{https://web.archive.org/web/20160628093159/http://www.nsd.uib.no/macrodataguide/country2.html?id=840\&c=United\%20States\%20of\%20America}
\url{https://www.cpds-data.org/}

Rubrics for grading (passing \textgreater{} 8 pts):

\begin{enumerate}
\def\labelenumi{\arabic{enumi}.}
\tightlist
\item
  Version control -- git -- all member make commits? (2 pts)
\item
  Reproducible Research adherance (2 pts)
\item
  Good Programing Practices followed (2 pts)
\item
  Overall project creativitity (2 pts)
\item
  Overall project execution (2 pts)
\end{enumerate}

\end{document}
