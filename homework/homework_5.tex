\PassOptionsToPackage{unicode=true}{hyperref} % options for packages loaded elsewhere
\PassOptionsToPackage{hyphens}{url}
%
\documentclass[]{article}
\usepackage{lmodern}
\usepackage{amssymb,amsmath}
\usepackage{ifxetex,ifluatex}
\usepackage{fixltx2e} % provides \textsubscript
\ifnum 0\ifxetex 1\fi\ifluatex 1\fi=0 % if pdftex
  \usepackage[T1]{fontenc}
  \usepackage[utf8]{inputenc}
  \usepackage{textcomp} % provides euro and other symbols
\else % if luatex or xelatex
  \usepackage{unicode-math}
  \defaultfontfeatures{Ligatures=TeX,Scale=MatchLowercase}
\fi
% use upquote if available, for straight quotes in verbatim environments
\IfFileExists{upquote.sty}{\usepackage{upquote}}{}
% use microtype if available
\IfFileExists{microtype.sty}{%
\usepackage[]{microtype}
\UseMicrotypeSet[protrusion]{basicmath} % disable protrusion for tt fonts
}{}
\IfFileExists{parskip.sty}{%
\usepackage{parskip}
}{% else
\setlength{\parindent}{0pt}
\setlength{\parskip}{6pt plus 2pt minus 1pt}
}
\usepackage{hyperref}
\hypersetup{
            pdftitle={Homework 5},
            pdfborder={0 0 0},
            breaklinks=true}
\urlstyle{same}  % don't use monospace font for urls
\usepackage[margin=1in]{geometry}
\usepackage{graphicx,grffile}
\makeatletter
\def\maxwidth{\ifdim\Gin@nat@width>\linewidth\linewidth\else\Gin@nat@width\fi}
\def\maxheight{\ifdim\Gin@nat@height>\textheight\textheight\else\Gin@nat@height\fi}
\makeatother
% Scale images if necessary, so that they will not overflow the page
% margins by default, and it is still possible to overwrite the defaults
% using explicit options in \includegraphics[width, height, ...]{}
\setkeys{Gin}{width=\maxwidth,height=\maxheight,keepaspectratio}
\setlength{\emergencystretch}{3em}  % prevent overfull lines
\providecommand{\tightlist}{%
  \setlength{\itemsep}{0pt}\setlength{\parskip}{0pt}}
\setcounter{secnumdepth}{0}
% Redefines (sub)paragraphs to behave more like sections
\ifx\paragraph\undefined\else
\let\oldparagraph\paragraph
\renewcommand{\paragraph}[1]{\oldparagraph{#1}\mbox{}}
\fi
\ifx\subparagraph\undefined\else
\let\oldsubparagraph\subparagraph
\renewcommand{\subparagraph}[1]{\oldsubparagraph{#1}\mbox{}}
\fi

% set default figure placement to htbp
\makeatletter
\def\fps@figure{htbp}
\makeatother

\usepackage{etoolbox}
\makeatletter
\providecommand{\subtitle}[1]{% add subtitle to \maketitle
  \apptocmd{\@title}{\par {\large #1 \par}}{}{}
}
\makeatother

\title{Homework 5}
\providecommand{\subtitle}[1]{}
\subtitle{Due Wednesday Nov 4, 2020}
\author{}
\date{\vspace{-2.5em}2020-10-12}

\begin{document}
\maketitle

For each assignment, turn in by the due date/time. Late assignments must
be arranged prior to submission. In every case, assignments are to be
typed neatly using proper English in Markdown.

This week, we spoke about Exploratory Data Analysis and plotting. To
begin the homework, we will as usual, start by loading, munging and
creating tidy data sets. In this homework, our goal is to create
informative (and perhaps pretty) plots showing features or perhaps
deficiencies in the data.

\hypertarget{problem-1}{%
\subsection{Problem 1}\label{problem-1}}

Work through the Swirl ``Exploratory\_Data\_Analysis'' lesson parts 1 -
10. If you need some review of ggplot, see the tutorial on
Rstudio.cloud.

\hypertarget{problem-2}{%
\subsection{Problem 2}\label{problem-2}}

Create a new R Markdown file within your local GitHub repo folder
(file--\textgreater{}new--\textgreater{}R Markdown--\textgreater{}save
as).

The filename should be: HW5\_lastname, i.e.~for me it would be
HW5\_Settlage

You will use this new R Markdown file to solve the following problems.

\hypertarget{problem-3}{%
\subsection{Problem 3}\label{problem-3}}

Using tidy concepts, get and clean the following data on education from
the World Bank.

\url{http://databank.worldbank.org/data/download/Edstats_csv.zip}

How many data points were there in the complete dataset? In your cleaned
dataset?

Choosing 2 countries, create a summary table of indicators for
comparison.

\hypertarget{problem-3-1}{%
\subsection{Problem 3}\label{problem-3-1}}

Using \emph{base} plotting functions, create a single figure that is
composed of the first two rows of plots from SAS's simple linear
regression diagnostics as shown here:
\url{https://support.sas.com/rnd/app/ODSGraphics/examples/reg.html}.
Demonstrate the plot using suitable data from problem 2.

\hypertarget{problem-4}{%
\subsection{Problem 4}\label{problem-4}}

Recreate the plot in problem 3 using ggplot2 functions. Note: there are
many extension libraries for ggplot, you will probably find an extension
to the ggplot2 functionality will do exactly what you want.

\hypertarget{problem-5}{%
\subsection{Problem 5}\label{problem-5}}

Finish this homework by pushing your changes to your repo.

\textbf{Only submit the .Rmd and .pdf solution files. Names should be
formatted HW8\_lastname\_firstname.Rmd and HW4\_lastname\_firstname.pdf}

\end{document}
